\

\section{\sofa \ Python API}

\subsection{The \textcode{Sofa} module}

The core of this plugin is the \textcode{Sofa} Python module available to python scripts from within the SofaPython components (they are not available outside Sofa environment, in the command-line \textcode{python} binary for example).

Therefore, each python script to be imbedded in \sofa should include the following line if it wants to interact with the \sofa framework:

\begin{code_python}
import Sofa
\end{code_python}

This module provides a wide range of methods and types, bound to essential \sofa \ framework features.

These can be used from python within scripts loaded by the components provided by this plugin.

%Module methods
\subsection{Module methods}

SofaPython provides several module methods, either for general purpose (use of factory, for example) or as convenient helper functions (node functions used directly with the tree root node).

\begin{itemize}
\item \textcode{createObject(BaseContext, BaseObjectDescription)}
\item \textcode{getObject(context,path)}
\item \textcode{getChildNode(node,path)}
\item \textcode{sendGUIMessage(msgType,msgValue)} 
\end{itemize}

%Class Hierarchy
\subsection{Types hierarchy}

The class hierarchy in the \textcode{Sofa} module is quite different from the C++ \sofa \ class hierarchy. Not all \sofa \ classes are bound in Python, and some levels in the hierarchy are skipped.

Despite the ability of Python to support multi-heritage, this feature has not been implemented in the \textcode{Sofa} module, for code simplicity.


\begin{figure}[htbp]
\begin{center}
\Tree [.Base [.BaseContext [.Context [.BaseNode [.Node GNode  ] ] ]  ] 
			[ .BaseObject BaseState ] ]
\Tree [.BaseObjectDescription ]
\Tree [.Vec3 ]
\Tree [.Data ]
\caption{Sofa python types hierarchy}
\label{default}
\end{center}
\end{figure}

\newpage

%Sofa.BaseObjectDescription
\subsubsection{Sofa.BaseObjectDescription}
Used to create objects with the Method \textcode{Sofa.createObject} :
\begin{code_python}
 desc = Sofa.BaseObjectDescription('particle','MechanicalObject')
 particle = Sofa.createObject(node,desc)
\end{code_python}

Methods :
\begin{itemize}
\item \textcode{setAttributes}
\item \textcode{getAttributes}
\end{itemize}
Constructor : 
\begin{itemize}
\item \textcode{BaseObjectDescription(name,type)}

\end{itemize}
Attributes :
\begin{itemize}
\item \textcode{name}
\end{itemize}

%Sofa.Vec3
\subsubsection{Sofa.Vec3}
Attributes :
\begin{itemize}
\item \textcode{x} 
\item \textcode{y}
\item \textcode{z}
\end{itemize}

%BASE
\subsubsection{Sofa.Base}

\subsubsection{Sofa.BaseObject}
\subsubsection{Sofa.BaseState}
Methods :
\begin{itemize}
\item resize(size)
\end{itemize}
\subsubsection{Sofa.BaseContext}
\subsubsection{Sofa.Context}
\subsubsection{Sofa.Node}
\subsubsection{Sofa.Gnode}

%DATA
\subsubsection{Sofa.Data}
Attributes :
\begin{itemize}
\item value
\item name
\end{itemize}
Methods :
\begin{itemize}
\item getValue(index)
\item setValue(index,value)
\end{itemize}

\subsubsection{Data members: the most important thing in \sofa Python API}

The most important class is \textcode{Sofa.Base}, and its associated method \textcode{findData}.
ALMOST EVERYTHING in \sofa is stored in Datas, and with the only \textcode{Sofa.Base.findData} method, almost everything is possible.
Through the \textcode{Sofa.Data class} (returned by \textcode{Sofa.Base.findData(name)} ) it is possible to read or write almost any object value, thus interact with the simulation in real-time.

This way, even if a specific component isn't bound to python, it's possible to access it by its \textcode{Sofa.Base} heritage.

\textcode{Data.value} attribute has a versatile behavior, depending on the Data type.

On read, \textcode{Data.value} can return either an integer, a float, a string, or even a list of one of these 3 types.
On write, you have to set EXACTLY the proper type, or you can set a string (same format as in the *.scn xml files).

Examples:

\begin{code_python}
print str(node.findData('gravity').value
\end{code_python}
will output the text conversion of a list of 3 floats :
\begin{code_bash}
[0.0, -9.81, 0.0]
\end{code_bash}

You can set it in two ways ; the native version :
\begin{code_python}
node.findData('gravity').value = [0.0, -9.81, 0.0]
\end{code_python}
or by the text version:
\begin{code_python}
node.findData('gravity').value = '0.0  -9.81 0.0'
\end{code_python}

Use of any other type will result in an error.
The following won't work for example :
\begin{code_python}
node.findData('gravity').value = 9.81
\end{code_python}

